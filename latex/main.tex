\documentclass{article}
\usepackage[utf8]{inputenc}

\usepackage{geometry}
\usepackage{bm}
\geometry{a4paper}
\usepackage{latexsym}
%\usepackage[dvips]{graphicx}
\usepackage{epsfig}
\usepackage{amsmath}
\usepackage{amsfonts}
\usepackage{amssymb}
\usepackage{eucal}
\usepackage{mathrsfs}
\usepackage{wasysym}
\usepackage{setspace}
\usepackage{float}
\usepackage{color}
\usepackage{rotating}
\usepackage{stmaryrd}
\usepackage{lineno}

\numberwithin{equation}{section}
\frenchspacing
%%
\usepackage{amsthm}


%%%%INSERITI ADESSO%%%%
\usepackage{amsmath}
\usepackage{amsfonts}
\usepackage{amssymb}
\usepackage{amsthm}
\usepackage{mathrsfs}
\usepackage{eucal}  
\theoremstyle{definition}
\usepackage{accents}
\usepackage{array}
\usepackage{cases}
\usepackage{graphicx}
\usepackage{booktabs}
\usepackage{caption}
\usepackage{cancel}
\usepackage{bbm}
\usepackage{subfig}
\usepackage{enumitem}
\usepackage{movie15}
 \usepackage{algorithm}
\usepackage{algpseudocode}
\usepackage{tabularx}
\usepackage{longtable}
 
% Font Management
\usepackage[T1]{fontenc}       % 8 bit font encoding: includes all accents
\usepackage{bm}                % alternative to \bs provided by package amsmath
\usepackage{bbm}               % alternative to \mathbb;  usage: \mathbbm{}
%\usepackage[mathscr]{eucal}    % alternative to \mathcal; usage: \mathcal{}
\usepackage{color}             % for text in colour
\usepackage{verbatim}          % environment for commenting out blocks of text
%\usepackage{exscale}           % needed to scale cmdx fonts
%\usepackage{ae,aecompl}        % see http://www.ctan.org/tex-archive/fonts/ae
%%%%%%%%%%%%%%%%%%


\theoremstyle{plain}
\newtheorem{thm}{Theorem}[section]
\newtheorem{lem}[thm]{Lemma}
\newtheorem{prop}[thm]{Proposizione}
\newtheorem*{cor}{Corollario}

\theoremstyle{definition}
\newtheorem{defn}{Definizione}[section]
\newtheorem{conj}{Congettura}[section]
\newtheorem{exmp}{Esempio}[section]

\theoremstyle{remark}
\newtheorem*{rem}{Osservazione}
\newtheorem*{note}{Nota}

\DeclareMathOperator*{\argmin}{argmin}
\DeclareMathOperator*{\argmax}{argmax}

\newcommand{\dom}{\mathrm{dom}}
\newcommand{\im}{\mathrm{im}}
\newcommand{\sign}{\mathrm{sign}}
\newcommand{\abs}{\mathrm{abs}}
\newcommand{\e}{\mathrm{exp}}

\setlength{\textwidth}{15 cm}
\setlength{\textheight}{23.5 cm}



%%%%%%%%%%%%%%%%%%%%%%%%%%%%%%%%%%%%%%%%%%%%%%%%%%%

\usepackage[utf8]{inputenc}
\usepackage[T1]{fontenc}
\usepackage{lmodern}
\usepackage[]{mdframed}
\usepackage{hyperref}
\hypersetup{%
    pdfpagemode={UseOutlines},
    bookmarksopen,
    pdfstartview={FitH},
    colorlinks,
    linkcolor={blue},
    citecolor={blue},
    urlcolor={blue}
  }

%%%%%%% use PDFLATEX 

\usepackage{lipsum} %to insert random text

\usepackage{geometry} %for the margins
\newcommand\fillin[1][4cm]{\makebox[#1]{\dotfill}} %for the dotted line in the frontispiace

\usepackage{dcolumn}
\newcolumntype{d}{D{.}{.}{-1} } %to vetical align numbers in tables, along the decimal dot

\usepackage{amsmath}



%%%%%%% Local definitions
\newtheorem{osservazione}{Osservazione}% Standard LaTeX
\newtheorem{observation}{Observation}% Standard LaTeX

\newcommand{\BR}{\mathscr{B}_{\mathrm{R}}}
\newcommand{\T}[2]{T_{#2}#1}
\newcommand{\cT}[2]{T_{#2}^{*}#1}
\newcommand{\pder}[2]{\frac{\partial #1}{\partial #2}}

				 
%%%%%%%%%%%%%%%%%%%%%%%%%%%%%%%%%%%%%%%%%%%%%%%%%
%
% Inserito il codice Matlab
%
\usepackage{listings}
\usepackage{hyperref}
\usepackage{xcolor}

\definecolor{codegreen}{rgb}{0,0.6,0}
\definecolor{codegray}{rgb}{0.5,0.5,0.5}
\definecolor{codepurple}{rgb}{0.58,0,0.82}
\definecolor{backcolour}{rgb}{0.95,0.95,0.92}

\lstdefinestyle{mystyle}{
    backgroundcolor=\color{backcolour},   
    commentstyle=\color{codegreen},
    keywordstyle=\color{magenta},
    numberstyle=\tiny\color{codegray},
    stringstyle=\color{codepurple},
    basicstyle=\ttfamily\footnotesize,
    breakatwhitespace=false,         
    breaklines=true,                 
    captionpos=b,                    
    keepspaces=true,                 
    numbers=left,                    
    numbersep=4pt,                  
    showspaces=false,                
    showstringspaces=false,
    showtabs=false,                  
    tabsize=2
}

\lstset{style=mystyle}



\title{Analisi tempo-frequenza e multiscala}
\author{Giulio Nenna - 292399}
\date{Homework 1 - Riconoscimento facciale attraverso "Eigenfaces"}
\begin{document}
\maketitle
\noindent Lo scopo dell'Homework è quello di implementare e testare l'algoritmo "\textit{Eigenfaces}" per il riconoscimento facciale attraverso uno script Python. Nell'elaborato verranno presentati gli strumenti teorici a supporto dell'algoritmo, frammenti di codice salienti utilizzati e risultati computazionali ottenuti.
\\
\\
\noindent Il dataset presenta 10 ritratti facciali per ciascuno dei 40 soggetti presenti al suo interno. Vengono utilizzate immagini in scala di grigio di dimensione \(m \times n \) pixels con \(m = 112\) e \(n = 92\). Alcuni esempi di immagini utilizzate sono mostrati in Figura \ref{subject_es} 

\begin{figure}[H]
  \centering
  \subfloat[1][Soggetto 1]{\includegraphics[scale = 0.37]{pictures/test_subject_1.pdf}}
  \subfloat[2][Soggetto 2]{\includegraphics[scale = 0.37]{pictures/test_subject_2.pdf}}
  \subfloat[3][Soggetto 31]{\includegraphics[scale = 0.37]{pictures/test_subject_31.pdf}}
  \caption{Alcuni esempi di immagini presenti nel dataset}
  \label{subject_es}
\end{figure}

\noindent L'algoritmo si distingue per la fase di \textbf{Training} e quella di \textbf{Testing}.
\section{Training phase}
Le immagini vengono inizialmente importate e vengono generati i dataset di \textit{Training} e \textit{testing}. In particolare nel dataset di training sono presenti le prime 6 immagini per ciascun soggetto mentre in quello di testing le restanti 4. Le immagini sono importate sotto forma di vettori \(\{f_1, f_2, \dots, f_N\}\) in \(\mathbb{R}^{mn}\). Il dataset di training contiene \(L = Np\) immagini, dove \(p\) è il rapporto tra dimensione del training set e dimensione del test set, nel nostro caso \(p = 0.6\).

\begin{lstlisting}[language=Python, caption= Data import]
faces_all = np.zeros([num_subjects*num_faces_per_subject, size])
faces_train = np.zeros([train_set_size, size])
faces_test = np.zeros([test_set_size, size])
print('Importing data...')
for i in range(num_subjects): #for each subject
    subject_faces = np.zeros([num_faces_per_subject, size]) #array containing all faces of the subject
    for j in range(num_faces_per_subject): #for each face
        f=path+'/s'+str(i+1)+'/'+str(j+1)+'.pgm' #compose filepath
        img = pgm.read_pgm(f) #read the file (2D array of shape m by n)
        img = img.reshape(size) #reshape the img into a 1D array
        subject_faces[j,:] = img 
    
    faces_all[i*num_faces_per_subject:(i+1)*num_faces_per_subject,:] = subject_faces
    faces_train[i*int(num_faces_per_subject*train_test_ratio):(i+1)*int(num_faces_per_subject*train_test_ratio),:] = \
        subject_faces[0:int(num_faces_per_subject*train_test_ratio),:]
    faces_test[i*int(num_faces_per_subject*(1-train_test_ratio)):(i+1)*int(num_faces_per_subject*(1-train_test_ratio)),:] = \
            subject_faces[int(num_faces_per_subject*train_test_ratio):num_faces_per_subject,:]
\end{lstlisting}

\noindent A questo punto viene calcolata la faccia media \(\tilde{f} = \frac{1}{L}  \sum \limits_{l=1}^{L} f_l\) e il dataset di training viene centrato rispetto alla faccia media, ottenendo nuovi vettori \(\phi_l = f_l - \tilde{f}\) con \(l \in \{1, \dots, L\}\).

\begin{lstlisting}[language=Python, caption= Centering data]
L =  faces_train.shape[0]
mean_face = faces_train.mean(axis = 0) #compute mean face
faces_train_center = faces_train-mean_face #centering the dataset
\end{lstlisting}

A questo punto l'idea è quella di trovare una particolare base ortonormale \(\{u_1, \dots,u_{mn}\}\) con \(u_i \in \mathbb{R}^{mn}, i=1,\dots mn\) tale per cui:
\begin{eqnarray*}
    \phi_l = \tilde{f} +  \sum \limits_{n\geq 1} \alpha_n u_n \\
|\alpha_1| \geq |\alpha_2| \geq, ..., \geq |\alpha_{mn}|
\end{eqnarray*}
La seconda condizione afferma che la base ortonormale che si sta cercando è tale per cui le componenti sono ordinate per importanza. Questo concetto è di fondamentale importanza per la pratica della riduzione dimensionale: se si riesce a trovare una base di rappresentazione delle immagini per cui le prime componenti sono più importanti delle ultime, allora sarà possibile rappresentare le immagini utilizzando il sottoinsieme delle prime componenti perdendo la minor quantità di informazioni possibile.
\\
Sia \(\alpha_1\) il coefficiente rispetto alla prima componente della base \(u_1\) per la generica immagine \(\phi_l\). 
\[
    \alpha_1 = \left\langle\phi_l, u_1 \right\rangle  
    \]
Sia \(\Phi^T\) la matrice contenente contenente per ciascuna riga le immagini di training:
\[\Phi^T = \begin{bmatrix}
    \phi_1^T \\
    \phi_2^T \\
    \vdots \\
    \phi_L^T
\end{bmatrix}
    \]

Allora il problema della ricerca della prima componente della base \(u_1\) che massimizza in media il modulo del relativo coefficiente per ogni immagine di training si formalizza come:
\begin{equation}
    \max_{\| x \| = 1 } \frac{1}{L}  \sum \limits_{l=1}^{L} | \langle \phi_l, x\rangle|^2 = \max_{\| x \| = 1 } \frac{1}{L} \| \Phi^T x \| ^2_2 = \max_{\| x \| = 1 } \left\langle \frac{1}{L}\Phi \Phi^T x, x \right\rangle = \lambda_1.
\end{equation}
In particolare \(\frac{1}{L}\Phi \Phi^T\) è la matrice di varianza-covarianza campionaria di \(\{\phi_1, ... \phi_L\}\) e \(\lambda_1\) è il suo più grande autovalore in modulo. \(u_1\) sarà pertanto l'autovettore relativo a \(\lambda_1\).
Per la ricerca della seconda componente il procedimento è lo stesso, imponendo l'ortogonalità rispetto alla prima componente:
\[
    \max_{\|x\| = 1, \, x \bot u_1 } \left\langle \frac{1}{L}\Phi \Phi^T x, x \right\rangle = \lambda_2.
\]
La ricerca della particolare base che stiamo cercando si riduce pertanto al problema del calcolo di autovalori e autovettori della matrice  \(\frac{1}{L}\Phi \Phi^T\).
\\
\\
Un'importante dettaglio implementativo è dato dalla dimensione della matrice \(\Phi\Phi^T \in \mathbb{R}^{mn \times mn}\). La quantità \(mn\)  è potenzialmente molto grande (nel nostro caso \(mn \sim 10^3\)), rendendo il calcolo degli autovalori e autovettori un'operazione computazionalmente molto onerosa. Questo problema può tuttavia essere aggirato se si considera che \(\text{Rank}(\Phi\Phi^T) \leq L \) e che gli autovalori della matrice \(\Phi^T \Phi\) sono gli stessi della matrice \(\Phi \Phi^T\). Inoltre:
\[ \Phi\Phi^T x = \lambda x \iff \Phi^T \Phi \Phi^T x = \lambda \Phi^T x
    \]
Pertanto se \(y = \Phi^T x\) è un autovettore di \(\Phi^T \Phi\), allora \(x = \Phi y\) sarà il corrispondente autovettore di \( \Phi\Phi^T \). Il calcolo degli autovettori può pertanto essere eseguito sulla matrice \(\Phi^T \Phi \in \mathbb{R}^{L \times L}\) con \(L \ll mn \) alleggerendo di molto i costi computazionali.
\\
\\
La particolare base che stiamo cercando sarà pertanto data da \(\{v_1, \dots, v_L\}\) dove 
\[ v_i = \Phi u_i \quad i = \{1, \dots, L\},\]
\[u_i \text{ autovettore di } \Phi^T\Phi \text{ relativo all'autovalore } \lambda_i ,\]
\[\lambda_1 \geq \lambda_2 \geq \dots \geq \lambda_L.\]
\section{Test Phase}
Siano \(\{g_1, g_2, \dots g_K\}\) le facce da utilizzare per il test set. Sono anch'esse vettori \( g_i \in \mathbb{R}^{mn}\) per \(i = 1, \dots K\). Sulla falsa riga di quanto fatto per il training, possiamo centrare ciascuna faccia rispetto alla faccia media e definire la matrice delle facce di test:
\[
\psi_i = g_i - \tilde{f} \quad i = 1,\dots K    
\]
\[\Psi^T = \begin{bmatrix}
    \psi_1^T \\
    \psi_2^T \\
    \vdots \\
    \psi_K^T
\end{bmatrix} \in \mathbb{R}^{K \times mn}
    \]
Grazie alla matrice \(V'\) calcolata in fase di training possiamo proiettare ciascuna faccia di test sullo spazio generato da \(V'\):
\[
\psi_i' = V'^T \psi_i \quad i = 1,\dots K
\]
\[
    \Psi'^T = \Psi^T V'
\]
Dove \(\psi'_i \in \mathbb{R}^{L'}\) e \(\Psi'^T \in \mathbb{R}^{K \times L'}\). \\
Una volta calcolate le proiezioni di ciascuna faccia sullo spazio generato dalle colonne di \(V'\), è possibile calcolarne la rappresentazione  rispetto alla base canonica di \(\mathbb{R}^{mn}\) utilizzando:
\[
\psi''_i = V' \psi_i \quad i=1, \dots K    
\]
\[
\Psi''^T = \Psi'^T V'^T    
\]
Dove \(\psi''_i \in \mathbb{R}^{mn}\) e \(\Psi''^T \in \mathbb{R}^{K \times mn}\). \\
A questo punto è possibile calcolare la distanza \(\epsilon\) tra ciascuna faccia di test e lo spazio generato da \(V'\), infatti:
\[
\epsilon_i = \| \psi_i - \psi''_i \| \quad i = 1,\dots K     
\]
\begin{lstlisting}[language=Python]
faces_test_centered = faces_test-mean_face
#project test faces onto eigenspace
faces_test_projected = faces_test_centered @ eigenfaces 
#project back onto face space
faces_test_projected_back = faces_test_projected @ eigenfaces.transpose() 
#compute distance from eigenspace for each face
distance_from_face_space = np.linalg.norm(faces_test_centered-faces_test_projected_back, axis=1) 
\end{lstlisting}
Dato quindi un threshold di accettazione \(\Theta\), è possibile verificare se una faccia appartiene al database utilizzato per generare l'autospazio controllando se \(\epsilon_i < \Theta\).
\newpage
\begin{mdframed}
    Un possibile metodo per calcolare il threshold \(\Theta\) potrebbe essere il seguente: si esclude un soggetto dal database sia di training che di test, si calcola la distanza della sua faccia dall'autospazio e si utilizza tale distanza come threshold \(\Theta\). Variazioni più accurate potrebbero coinvolgere l'utilizzo di più facce e più soggetti non presenti nel database per il calcolo di \(\Theta\). Il \href{https://github.com/giulionenna/Facial_Recognition/blob/main/facial_recognition.py}{codice completo} utilizzato implementa un calcolo approssimato di \(\Theta\) escludendo le 10 facce dell'ultimo soggetto dal dataset.
\end{mdframed}
Verificato che una faccia sia presente nel dataset sarà quindi possibile predirre a quale soggetto corrisponde la faccia \(\psi'\) individuando la faccia \(\phi'^*\) più vicina nell'autospazio. In particolare se \(\phi'^*\) è la faccia appartenente al soggetto \(s\) più vicina a \(\psi'\) nell'autospazio, allora l'algoritmo assegnerà alla faccia \(\psi'\) il soggetto \(s\) come predizione. 
\begin{lstlisting}[language = Python]
for i in range(test_set_size):#for each face in the test set
    face = faces_test_projected[i]
    #if the face is too far from eigenspace
    if distance_from_face_space[i]>acceptance_threshold: 
        continue #skip face
    #compute distance of the face with every other face in the eigenspace
    dist = np.linalg.norm(faces_train_projected - face, axis=1) 
    #find the nearest face
    idx = np.argmin(dist) 
    #predict using idx of the nearest face
    predicted[i] = idx // int(num_faces_per_subject*train_test_ratio) 
\end{lstlisting}

%\bibliographystyle{plain} % We choose the "plain" reference style
%\bibliography{refs} % Entries are in the refs.bib file


\end{document}